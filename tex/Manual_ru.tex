\documentclass[a4paper,12pt]{jpconf}

\usepackage{cmap}
\usepackage[T2A]{fontenc}
\usepackage[utf8]{inputenc}
\usepackage[english,russian]{babel}

%\usepackage{makeidx}
\usepackage{amsmath,amssymb,amsthm,amsfonts,latexsym,mathtext}
\usepackage[pdftex]{graphicx}
\usepackage{float}
%\usepackage{citesort}
\usepackage{subfigure}
\graphicspath{{fig/}{pict3/}}
\usepackage{ifpdf}
\ifpdf\usepackage{epstopdf}\fi

\usepackage{bm,mathtools}
\usepackage{geometry}
\geometry{top=2cm}
\geometry{bottom=2.5cm}
\geometry{left=2.5cm}
\geometry{right=2cm}

\bibliographystyle{iopart-num}

\newcommand{\be}{\begin{equation}}
	\newcommand{\ee}{\end{equation}}
\newcommand{\ba}{\begin{aligned}}
	\newcommand{\ea}{\end{aligned}}

\def\lsim{\;\raise0.3ex\hbox{$<$\kern-0.75em\raise-1.1ex\hbox{$\sim$}}\;}
\def\gsim{\;\raise0.3ex\hbox{$>$\kern-0.75em\raise-1.1ex\hbox{$\sim$}}\;}
\begin{document}
	\title{Преобразование функции распределения}
	\section{Преобразование функции распределения фотонов}
	Функция распределения фотонов задана в сферических координатах $n_{ph}(\epsilon,\mu,\phi)$. Рассмотрим переход в систему отсчета, движущуюся в направлении оси $z$ с лоренц-фактором $\gamma = 1/\sqrt{1-\beta^2}$. Количество частиц в элементе фазового пространства $N$ - инвариант.
	\begin{equation}
		N = n_{ph}(\epsilon,\mu,\phi)d\epsilon d\mu d\phi dV = n'_{ph}(\epsilon',\mu',\phi')d\epsilon' d\mu' d\phi' dV'
	\end{equation}
	Рассмотрим преобразование вектора четырех-импульса. Поперечные компоненты не изменяются, а временная и продольная меняются следющим образом, учитывая что $p_z = \mu \epsilon$:
	\begin{equation}\label{lorentz_ph}
		\left(\begin{array}{c}
			\epsilon'\\
			\mu'\epsilon'
		\end{array}
		\right)
		= \left(
		\begin{array}{cc}
			\gamma & -\beta\gamma\\
			-\beta\gamma & \gamma
		\end{array}
		\right)
		\times
		\left(\begin{array}{c}
			\epsilon\\
			\mu\epsilon
		\end{array}
		\right)
	\end{equation}
	Из первой строчки матрицы получаем уравнение для допплеровского сдвига энергии
	\begin{equation}\label{doppler_ph}
		\epsilon'=\gamma(1-\mu\beta)\epsilon
	\end{equation}
	Вычислим производные новой энергии по старым координатам
	\begin{equation}
		\frac{d\epsilon'}{d\epsilon}=\gamma(1-\mu\beta)
	\end{equation}
	\begin{equation}
		\frac{d\epsilon'}{d\mu}=-\gamma\beta\epsilon
	\end{equation}
	Из второй строчки матрицы получаем $\mu'\epsilon'=-\beta\gamma\epsilon+\gamma\mu\epsilon$. Подставив значение $\epsilon'$ из \ref{doppler_ph} и сократив $\epsilon$ получим уравнение аберрации света
	\begin{equation}\label{aberration_ph}
		\mu'=\frac{\mu-\beta}{1-\mu\beta}
	\end{equation}
	Заметим, что угол наклона луча в новой системе не зависит от энергии в старой системе. Вычислим частноую производную $\frac{d\mu'}{d\mu}$
	\begin{equation}
		\frac{d\mu'}{d\mu}=\frac{d}{d\mu}\frac{1}{\beta}\frac{\beta\mu-1+1-\beta^2}{1-\mu\beta}=\frac{d}{d\mu}\frac{1}{\beta}\frac{1-\beta^2}{1-\mu\beta}=\frac{1-\beta^2}{(1-\mu\beta)^2}=\frac{1}{\gamma^2(1-\mu\beta)^2}
	\end{equation}
	Азиметальный угол не зависит от системы отсчета $\phi' = \phi$. Преобразование элемента объема описывается выражением $\frac{dV'}{dV} = \frac{\epsilon}{\epsilon'}$ см. ЛЛ Т2 параграф 10, вот только там используется переход в собственную систему. То есть
	\begin{equation}
		\frac{dV'}{dV} = \frac{1}{\gamma(1-\mu\beta)}
	\end{equation}
	Матрица якоби преобразования координат выглядит следующим образом
	\begin{equation}
		J=\left(
		\begin{array}{cccc}
			\frac{d\epsilon'}{d\epsilon} & \frac{d\epsilon'}{d\mu}& 0 & 0\\
			0 & \frac{d\mu'}{d\mu} & 0 & 0\\
			0 & 0 & 1 & 0\\
			0 & \frac{dV'}{d\mu} & 0 & \frac{dV'}{dV}
		\end{array}
		\right)
	\end{equation}
	При такой матрице якобиан, к счастью, равен произведению диагональных членов
	\begin{equation}\label{jacobian_ph}
		\frac{D(\epsilon',\mu',\phi',V')}{D(\epsilon,\mu,\phi,V)}=\frac{d\epsilon'}{d\epsilon}\frac{d\mu'}{d\mu}\frac{dV'}{dV}=\gamma(1-\mu\beta)\frac{1}{\gamma^2(1-\mu\beta)^2}\frac{1}{\gamma(1-\mu\beta)}=\frac{1}{\gamma^2(1-\mu\beta)^2}
	\end{equation}
	И в итоге функция распределения фотонов преобразуется с помощью деления на вычисленный якобиан
	\begin{equation}\label{distribution_ph}
		n'_{ph}(\epsilon',\mu',\phi') = \frac{n_{ph}(\epsilon,\mu,\phi)}{\frac{D(\epsilon',\mu',\phi',V')}{D(\epsilon,\mu,\phi,V)}}=\gamma^2(1-\mu\beta)^2 n_{ph}(\epsilon,\mu,\phi)
	\end{equation}
	\section{Комптоновское рассеяние}
	Рассмотрим рассеяние фотонов на одном электроне. Сечение Клейна-Нишины в системе покоя электрона равно
	\begin{equation}
		\frac{d\sigma}{d\epsilon_1'd\Omega_1'}=\frac{{r_e}^2}{2}\left(\frac{\epsilon_1'}{\epsilon_0'}\right)^2\left(\frac{\epsilon_1'}{\epsilon_0'}+\frac{\epsilon_0'}{\epsilon_1'}-sin^2\Theta'\right)
	\end{equation}
	Где $r_e$ - классический радиус электрона, $\epsilon_0'$ и $\epsilon_1'$ - энергии начального и конечного фотона, соответственно, $\Theta'$ - угол между начальным и конечным фотоном. Штрихованные индексы относятся к системе отсчета электрона.
	Число фотонов, 
\end{document}
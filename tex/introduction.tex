\chapter*{Введение}					
\addcontentsline{toc}{chapter}{Введение}	%добавляем в оглавление
FAINA - чилсенный код, предназначенный для расчетов различных видов электромагнитного излучения от астрофизических источников. Код написан на языке C++ с использованием только стандартной библиотеки. В текущей версии реализованы следующие виды излучения: синхротронное излучение, излучение за счет обратного комптоновского рассеяния, излучение распада пионо в результате свободно-свободных столкновений протонов а так же тормозное излучение. FAINA позволяет вычислять наблюдаемые потоки от источников с заданными параметрами, а так же вычислять параметры источников фитированием наблюдаемых данных расчетными.

\section*{Установка и запуск}
\addcontentsline{toc}{section}{Установка и запуск}	%добавляем в оглавление

Текущая версия кода доступна на github https://github.com/VadimRomansky/Faina. Скачайте архив и разархивируйте его в директорию Faina. 
\subsection{Windows}
Для работы с кодом и его запуска необходимо использовать Microsoft Visual Studio и открыть с помощью неё файл Faina.sin, содержащийся в корневой директории кода. Работоспособность проверялась на версии Visual Studio 2022.
\subsection{Linux}
Для запуска FAINA в операционной системе Linux предусмотрены два варианта. Рекомендуется использовать среду разработки QtCreator и открыть с помощьб неё проектный файл Faina.pro, содержащийся в корневой дирректории кода. 

Так же возможна непосредственная компиляция и запуск из терминала, с помощью комманд
\begin{lstlisting}[language=bash]
	$ g++ -o faina *.cpp
	$ ./faina
\end{lstlisting}




%%% Поля и разметка страницы %%%
\usepackage{lscape}	    % Для включения альбомных страниц
\usepackage{geometry}	% Для последующего задания полей
%\usepackage{subfigure}  % Для создания рисунков с несколькими панелями

%%% Кодировки и шрифты %%%
\usepackage{cmap}			% Улучшенный поиск русских слов в полученном pdf-файле
\usepackage[T2A]{fontenc}		% Поддержка русских букв
\usepackage[utf8]{inputenc}		% Кодировка utf8
\usepackage[english, russian]{babel}	% Языки: русский, английский
%\usepackage{pscyr}			% Красивые русские шрифты

%%% Математические пакеты %%%
\usepackage{amsthm,amsfonts,amsmath,amssymb,amscd} % Математические дополнения от AMS

%%% Оформление абзацев %%%
\usepackage{indentfirst} % Красная строка

%%% Цвета %%%
\usepackage[usenames]{color}
\usepackage{color}
\usepackage{colortbl}

%%% Таблицы %%%                        
%\usepackage{longtable}			% Длинные таблицы
%\usepackage{multirow,makecell,array}	% Улучшенное форматирование таблиц

%%% Общее форматирование
\usepackage[singlelinecheck=off]{caption}	% Многострочные подписи
%\usepackage{soul}					% Поддержка переносоустойчивых подчёркиваний и зачёркиваний

%%% Библиография %%%
\usepackage[numbers]{natbib}
%\usepackage{cite} % Красивые ссылки на литературу
%\usepackage{csquotes}
%\usepackage[autolang=other,       
%            bibencoding=utf8,
%            sorting=none, % Name,Title,Year или sorting=none.
%            maxbibnames=3, % Максимальное число авторов в списке литературы.
%            minbibnames=2, % Число авторов, отображаемое при сокращении.
%            style=gost-numeric,
%            backend=biber]{biblatex}


\usepackage{placeins}

%%% Гиперссылки %%%
\usepackage[linktocpage=true,plainpages=false,pdfpagelabels=false,unicode=true]{hyperref}

%%% Изображения %%%
\usepackage{graphicx} % Подключаем пакет работы с графикой
%\usepackage{epstopdf}

%%% Оглавление %%%
\usepackage{tocloft}

\usepackage{titlesec}

%\usepackage{minted} %консольные комманды
\usepackage{listings}
\usepackage{tabularx}

\chapter{Оптимизация параметров}\label{optimization}
Код FAINA позволяет не только расчитывать излучение заданных источников, но и фитировать наблюдательные данные модельными, подбирая необходимые параметры. Реализованы методы оптимизации, пригодные для произвольного числа параметров и широкого класса моделей источников. В качестве целевой функции используется взвешенная сумма квадратов отклонений по всем наблюдательным точкам $f = \sum \frac{(F_i - F_{obs,i})^2}{\sigma_i^2}$, где $F_i$ - расчетная спектральная плотность потока излучения, $F_{obs,i}$ - наблюдаемая спектральная плотность потока излучения, $\sigma_i$ - её погрешность. В текущей версии учитывается лишь погрешность измеряемого потока, ширина бина и неопределенность энергии, на которой принят сигнал не учитываются.

Реализованные методы оптимизации делятся на два типа - те, которые рассматривают излучение в один момент времени, либо постоянные во времени, и те, которые учитывают эволюцию источников и используют наблюдения в разные моменты времени. В последнем случае пользователю необходимо самостоятельно указывать, как меняются параметры источника со временем, см. раздел \ref{timeDependentSource}.

\section{Фитирование источников, не зависящих от времени}
Для фитирования постоянных во времени кривых блеска предназначен абстрактный класс RadiationOptimizer. В нем определена виртуальныя функция optimize(double* vector, bool* optPar, double* energy, double* observedFlux, double* observedError, int Ne, RadiationSource* source), которая и производит процесс оптимизации. Входными параметрами являются: vector - массив подбираемых параметров, в который будет записан результат работы программы, optPar - массив булевских переменных, определяющих оптимизировать соответствующий параметр, или считать его фиксированным, energy - массив энергий, на которых производились наблюдения, observedFlux - соответствующие наблюдаемые потоки в единицах $\text{см}^{-2}\text{с}^{-1}$, Ne - количество наблюдательных точек, source - источник излучения. Функция изменения параметров источника source->resetParameters, описанная в разделе \ref{sourcesSection}, должна быть согласована с массивом оптимизируемых параметров vector, так как в процессе оптимизации он будет передаваться в нее в качестве аргумента.

В коде реализованы два наследника класса RadiationOptimazer: GridEnumRadiationOptimizer - производящий поиск минимума простым перебором по сетке параметров с заданным количеством распределенных равномерно логарифмически точек, и GradientDescentRadiationOptimizer - в котором минимум находится методом градиентного спуска. Эти два класса полезно использовать совместно, используя результат работы первого как начальную точку для второго. Схема насследования классов оптимизаторов показана на рисунке \ref{radiationOptimizer}, а список их публичных методов приведен в Таблице \ref{RadiationOptimizerMethods}. Реализованные методы оптимизации применимы для всех описанных выше типов источников и видов электромагнитного излучения.
\begin{figure}
	\centering
	\includegraphics[width=11.5 cm]{./fig/radiationOptimizer.png} 
	\caption{Схема наследования классов оптимизаторов}
	\label{radiationOptimizer}
\end{figure}

\begin{small}
	\topcaption{Публичные методы классов оптимизаторов параметров источников }
	\label{RadiationOptimizerMethods}
	\begin{xtabular}{|p{0.41\textwidth}|p{0.59\textwidth}|}
		\hline
		\textbf{RadiationOptimizer} & абстрактный класс для оптимизации параметров источника \\
		\hline
		evaluateOptimizationFunction( const double* vector, double* energy, double* observedFlux, double* observedError, int Ne, RadiationSource* source) & вычисляет целевую функцию - взвешенную сумму квадратов ошибок во всех наблюдательных точках $f = \sum \frac{(F_i - F_{obs,i})^2}{\sigma_i^2}$, где $F_i$ - расчетная спектральная плотность потока излучения, $F_{obs,i}$ - наблюдаемая спектральная плотность потока излучения, $\sigma_i$ - её погрешность\\
		\hline
		optimize( double* vector, bool* optPar, double* energy, double* observedFlux, double* observedError, int Ne, RadiationSource* source) & функция, осуществляющая оптимизацию, принимает на вход массив подбираемых параметров, в который будет записан результат, массив булевских переменных, определяющих оптимизировать соответствующий параметр, или считать его фиксированным, массив энергий, на которых производились наблюдения, соответствующие наблюдаемые потоки в единицах $\text{см}^{-2}\text{с}^{-1}$, погрешности измерения потоков, количество наблюдательных точек, и источник излучения.\\
		\hline
		optimize( double* vector, bool* optPar, double* energy, double* observedFlux, int Ne, RadiationSource* source) & функция, осуществляющая оптимизацию, в случае не заданных наблюдательных ошибок. В таком случае ошибки у всех точек считаются равными елинице и веса всех ошибок в целевой функции оказываются равными\\
		\hline
		\textbf{GridEnumRadiationOptimizer} & класс предназначенный для оптимизации параметров с помощью перебора по сетке\\
		\hline
		GridEnumRadiationOptimizer( RadiationEvaluator* evaluator, const double* minParameters, const double* maxParameters, int Nparams, const int* Npoints) & конструктор, создает экземпляр класса с указанным вычислителем излучения, минимальными и максимальными значениями оптимизируемых параметров, количеством этих параметров и массивом с количеством перебираемых точек по каждому параметру. При переборе точки будут распределены логарифмически равномерно по оси.\\
		\hline
		\textbf{GradientDescentRadiationOptimizer} & класс, предназначенный для оптимизации параметров методом градиентного спуска\\
		\hline
		GradientDescentRadiationOptimizer( RadiationEvaluator* evaluator, const double* minParameters, const double* maxParameters, int Nparams, int Niterations) & конструктор, создает экземпляр класса с указанным вычислителем излучения, минимальными и максимальными значеними оптимизируемых параметров, количеством этих параметров и максимальным количеством итераций градиентного спуска\\
		\hline		
	\end{xtabular}
\end{small}

Пример фитирования параметров источника по наблюдательным данным приведен в функции fitCSS161010withPowerLawDistribition в файле examples.cpp. Следуя авторам работы \cite{Coppejans2020} произведем расчет синхротронного излучения источника с учетом самопоглощения, считая функцию распределения электронов чисто степенной с показателем 3.6. Но мы не будем накладывать дополнительную связь на параметры и предполагать равенство распределения энергии между магнитным полем и ускоренными частицами, вместо этого магнитное поле и концентрация электронов будут независимыми параметрами.

Подберем параметры Быстрого Оптического Голубого Транзиента CSS161010 на 98 день после вспышки на основе радиоизлучения. Зададим параметры источника на основе дынных статьи \cite{Coppejans2020}, которые будут использоваться в качестве начального приближения, а так же расстояние до него.
\begin{lstlisting}[language=c++]
    double electronConcentration = 25;
    double B = 0.6;
    double R = 1.4E17;
    double fraction = 0.5;
    const double distance = 150 * 1E6 * parsec;
\end{lstlisting}
Далее зададим степенное распределение электронов, с показателем 3.6 и источник в форме плоского диска, перпендикулярного лучу зрения, и вычислитель синхротронного излучения.
\begin{lstlisting}[language=c++]
    double Emin = me_c2;
    double Emax = 10000 * me_c2;
    double index = 3.6;
	
    SynchrotronEvaluator* synchrotronEvaluator = new 
	    SynchrotronEvaluator(200, Emin, Emax);
    MassiveParticlePowerLawDistribution* electrons = new 
	    MassiveParticlePowerLawDistribution(massElectron, index, 
	    Emin, electronConcentration);
    SimpleFlatSource* source = new 
	    SimpleFlatSource(electrons, B, 1.0, R, fraction*R, distance);
\end{lstlisting}
Теперь определим вектор оптимизируемых параметров - это размер, магнитное поле, концентрация электронов и доля толщины, показывающая какю долю от радиуса диска составляет его толщина. И именно такие параметры ожидает функция resetParameters у источника SimpleFlatSource. Так же нужно указать минимальные и максимальные значения параметров, которые ограничат область поиска. Максимальные значения так же будут использоваться как константы нормировки.
\begin{lstlisting}[language=c++]
    const int Nparams = 4;
    double minParameters[Nparams] = { 1E17, 0.01, 0.5, 0.1 };
    double maxParameters[Nparams] = { 2E17, 10, 200, 1.0 };
    double vector[Nparams] = { R, B, electronConcentration, fraction};
    for (int i = 0; i < Nparams; ++i) {
	    vector[i] = vector[i] / maxParameters[i];
    }
\end{lstlisting}
Зададим наблюдательные данные, которые и будем фитировать. Обратите внимание, что частоты нужно перевести в энергии, а спектральную плотность потока - в энергетическую (в единицы $\text{см}^{-2}\text{с}^{-1}$).
\begin{lstlisting}[language=c++]
    const int Nenergy1 = 4;
    double energy1[Nenergy1] = { 1.5E9*hplank, 3.0E9 * hplank, 
    	6.1E9 * hplank, 9.8E9 * hplank };
    double observedFlux[Nenergy1] = { 1.5/(hplank*1E26), 
    	4.3/(hplank*1E26), 6.1/(hplank*1E26), 4.2 /(hplank*1E26)};
    double observedError[Nenergy1] = { 0.1 / (hplank * 1E26), 
    	0.2/(hplank*1E26), 0.3/(hplank*1E26), 0.2/(hplank*1E26)};
\end{lstlisting}
Далее создадим два оптимизатора - действющий перебором и градиентым спуском, и применим их последовательно. Так же укажем количество точек для перебора и то, что оптимизируем все параметры.
\begin{lstlisting}[language=c++]
    bool optPar[Nparams] = { true, true, true, true };
    int Niterations = 20;
    int Npoints[Nparams] = { 10,10,10,10 };
    
    RadiationOptimizer* enumOptimizer = new GridEnumRadiationOptimizer(
        synchrotronEvaluator,minParameters,maxParameters,Nparams,Npoints);
    RadiationOptimizer* gradientOptimizer = new 
        GradientDescentRadiationOptimizer(synchrotronEvaluator, 
        minParameters, maxParameters, Nparams, Niterations);
\end{lstlisting}
Применим функцию optimize у последовательно у обоих оптимизаторов. Сначала перебором найдем начальное приближение, потом уточним результат с помощью градиентного спуска, и изменим параметры источника на оптимальные
\begin{lstlisting}[language=c++]
    enumOptimizer->optimize(vector, optPar, energy1, observedFlux, 
        observedError, Nenergy1, source);
    gradientOptimizer->optimize(vector, optPar, energy1, observedFlux, 
        observedError, Nenergy1, source);
    source->resetParameters(vector, maxParameters);
\end{lstlisting}
Полученные в результате оптимизации парметры источника равны: радиус диска $R = 1.8\times10^17 \text{ см}$, магнитное поле $B = 1.6 \text{ Гс}$, концентрация электронов $n = 2.3 \text{ см}^{-3}$, доля толщины $fraction = 0.54 $. Значение целевой функции $f \approx 50$. Модельный спектр излучения  с данными параметрами и наблюдательные данные изображены на рисунке \ref{synchrotron1}.
\begin{figure}
	\centering
	\includegraphics[width=12.5 cm]{./fig/synchrotron1.png} 
	\caption{Наблюдаемый и расчетный спектр радиоизлучения объекта CSS161010 на 98 день после вспышки}
	\label{synchrotron1}
\end{figure}
\section{Фитирование источников, зависящих от времени}
Для фитирования постоянных во времени кривых блеска изменяющихся во времени предназначен абстрактный класс RadiationTimeOptimizer. В нем определена виртуальныя функция optimize(double* vector, bool* optPar, double** energy, double** observedFLux, double** observedError, int* Ne, int Ntimes, double* times, RadiationTimeDependentSource* source), которая и выполняет оптимизацию. Так же как и в случае не зависящей от времени оптимизации она принимает на вход массив параметров, массив булевских переменных, и наблюдательные данные. Но наблюдательные данные теперь представляют собой двумерные массивы (причем количество заданных точек в разные моменты времени так же может быть разным). Так же нужно указать количество серий измерений во времени и соответствующие им времена. Исследуемый источник должен относиться к классу зависящих от времени источников.

Как и ранее, в коде реализованы два наследника класса RadiationTimeOptimazer: GridEnumRadiationTimeOptimizer - для поиска минимума перебором, и GradientDescentRadiationTimeOptimizer - в котором минимум находится методом градиентного спуска. Схема насследования классов оптимизаторов показана на рисунке \ref{radiationOptimizerTime}, а список их публичных методов приведен в Таблице \ref{RadiationTimeOptimizerMethods}.
\begin{figure}
	\centering
	\includegraphics[width=10.5 cm]{./fig/radiationOptimizerTime.png} 
	\caption{Схема наследования классов оптимизаторов, учитывающих переменность источников}
	\label{radiationOptimizerTime}
\end{figure}

\begin{small}
	\topcaption{Публичные методы классов оптимизаторов параметров переменных во времени источников }
	\label{RadiationTimeOptimizerMethods}
	\begin{xtabular}{|p{0.5\textwidth}|p{0.5\textwidth}|}
		\hline
		\textbf{RadiationTimeOptimizer} & абстрактный класс, предназначенный для оптимизации параметров зависящих от времени источников\\
		\hline
		evaluateOptimizationFunction( const double* vector, double** energy, double** observedFlux, double** observedError, int* Ne, int Ntimes, double* times, RadiationTimeDependentSource* source) & вычисляет целевую функцию - взвешенную сумму квадратов ошибок во всех наблюдательных точках во все моменты времени $f = \sum \frac{(F_i - F_{obs,i})^2}{\sigma_i^2}$, где $F_i$ - расчетная спектральная плотность потока излучения, $F_{obs,i}$ - наблюдаемая спектральная плотность потока излучения, $\sigma_i$ - её погрешность\\
		\hline
		optimize (double* vector, bool* optPar, double** energy, double** observedFLux, double** observedError, int* Ne, int Ntimes, double* times, RadiationTimeDependentSource* source) & функция, осуществляющая оптимизацию, принимает на вход массив подбираемых параметров, в который будет записан результат, массив булевских переменных, определяющих оптимизировать соответствующий параметр, или считать его фиксированным, двумерные массивы энергий, на которых производились наблюдения, соответствующих энергиям наблюдаемых потоков в единицах $\text{см}^{-2}\text{с}^{-1}$, погрешности наблюдаемых потоков, массив количества наблюдательных точек в каждый момент времени, количество моментов времени, когда производились измерения, соответствующие времена и переменный источник излучения.\\
		\hline
		optimize(double* vector, bool* optPar, double** energy, double** observedFlux, int* Ne, int Ntimes, double* times, RadiationTimeDependentSource* source) & функция, осуществляющая оптимизацию, в случае не заданных наблюдательных ошибок. В таком случае ошибки у всех точек считаются равными елинице и веса всех ошибок в целевой функции оказываются равными\\
		\hline
		\textbf{GridEnumRadiationTimeOptimizer} & класс, предназначенный для оптимизации параметров переменных источников с помощью перебора по сетке\\
		\hline
		GridEnumRadiationTimeOptimizer( RadiationEvaluator* evaluator, const double* minParameters, const double* maxParameters, int Nparams, const int* Npoints) & конструктор, создает экземпляр класса с указанным вычислителем излучения, минимальными и максимальными значениями оптимизируемых параметров, количеством этих параметров и массивом с количеством перебираемых точек по каждому параметру. При переборе точки будут распределены логарифмически равномерно.\\
		\hline
		\textbf{GradientDescentRadiationTimeOptimizer} & класс, предназначенный для оптимизации параметров переменных источников методом градиентного спуска\\
		\hline
		GradientDescentRadiationTimeOptimizer( RadiationEvaluator* evaluator, const double* minParameters, const double* maxParameters, int Nparams, int Niterations) & конструктор, создает экземпляр класса с указанным вычислителем излучения, минимальными и максимальными значеними оптимизируемых параметров, количеством этих параметров и максимальным количеством итераций градиентного спуска\\
		\hline		
\end{xtabular}
\end{small}

Пример фитирования параметров источника по наблюдательным данным приведен в функции fitTimeDependentCSS161010() в файле examples.cpp. Подберем параметры Быстрого Оптического Голубого Транзиента CSS161010 на основе наблюдений радиоизлучения, проведенных на 98, 162, 357 день после вспышки.  Расчет синхротронного излучения учитывает самопоглощение и раширение источника. Источник будем считать шаровым слоем с однородной плотностью и однороным магнитным полем, направленным перпендикулярно лучу зрения. Функцию распределения излучающих электронов возьмем на основе Particle-in-Cell расчетов для ударной волны со скоростью 0.3c,как сделано в работе \cite{BykovUniverse}. Учтена зависимость функции распределения от угла между магнитным полем и направлением распространения ударной волны.

Подберем параметры Быстрого Оптического Голубого Транзиента CSS161010 на основе наблюдений радиоизлучения, проведенных на 98, 162, 357 день после вспышки.  Зададим сначала массивы наблюдательных точек, переведя при этом из единиц герцы и милиянские в эрги и $\text{см}^{-2} \text{c}^{-1}$
\begin{lstlisting}[language=c++]
const double cssx1[4] = 
{1.5*hplank*1E9, 3.0*hplank*1E9, 6.1*hplank*1E9, 9.87*hplank*1E9};
const double cssy1[4] =  {1.5/(hplank*1E26), 
4.3/(hplank*1E26), 6.1/(hplank*1E26), 4.2/(hplank*1E26)};
const double cssError1[4] = {0.1/(hplank*1E26), 
0.2/(hplank*1E26), 0.3/(hplank*1E26), 0.2/(hplank*1E26)};
	
const double cssx2[4] = 
{2.94*hplank*1E9, 6.1*hplank*1E9, 9.74*hplank*1E9, 22.0*hplank*1E9};
const double cssy2[4] = {2.9/(hplank*1E26), 
2.3/(hplank*1E26), 1.74/(hplank*1E26), 0.56/(hplank*1E26)};
const double cssError2[4] = {0.2/(hplank*1E26), 
0.1/(hplank*1E26), 0.09/(hplank*1E26), 0.03/(hplank*1E26)};
	
const double cssx3[6] = {0.33*hplank*1E9, 0.61*hplank*1E9, 
1.5*hplank*1E9, 3.0*hplank*1E9, 6.05*hplank*1E9, 10.0*hplank*1E9};
const double cssy3[6] = {0.375/(hplank*1E26), 0.79/(hplank*1E26), 
0.27/(hplank*1E26),0.17/(hplank*1E26),0.07/(hplank*1E26),0.32/(hplank*1E27)};
const double cssError3[6] = {0.375/(hplank*1E26), 0.09/(hplank*1E26),
0.07/(hplank*1E26),0.03/(hplank*1E26),0.01/(hplank*1E26),0.8/(hplank * 1E28)};
\end{lstlisting}

Определим моменты времени инаблюдений и соответствующие им количества точек

\begin{lstlisting}[language=c++]
const int Ntimes = 3;
double times[Ntimes] = { 99 * 24 * 3600, 162 * 24 * 3600, 357 * 24 * 3600 };
int Nenergy[Ntimes];
Nenergy[0] = 4;
Nenergy[1] = 4;
Nenergy[2] = 6;
\end{lstlisting}
Создадим и инициализируем необходимые массивы с наблюдательными данными
\begin{lstlisting}[language=c++]
double** energy = new double* [Ntimes];
double** F = new double* [Ntimes];
double** Error = new double* [Ntimes];
for (int m = 0; m < Ntimes; ++m) {
	energy[m] = new double[Nenergy[m]];
	F[m] = new double[Nenergy[m]];
	Error[m] = new double[Nenergy[m]];
}

for (int i = 0; i < Nenergy[0]; ++i) {
	energy[0][i] = cssx1[i];
	F[0][i] = cssy1[i];
	Error[0][i] = cssError1[i];
}

for (int i = 0; i < Nenergy[1]; ++i) {
	energy[1][i] = cssx2[i];
	F[1][i] = cssy2[i];
	Error[1][i] = cssError2[i];
}

for (int i = 0; i < Nenergy[2]; ++i) {
	energy[2][i] = cssx3[i];
	F[2][i] = cssy3[i];
	Error[2][i] = cssError3[i];
}
\end{lstlisting}

Зададим физические параметры источника (или их начальные приближения) - расстояние, размер, концентрацию, магнитное поле, долю толщины шара, занятую излучающим веществом, скорость расширения и магнетизацию.

\begin{lstlisting}[language=c++]
const double distance = 150 * 1E6 * parsec;
double rmax = 1.3E17;
double electronConcentration = 150;
double B = 0.6;
double widthFraction = 0.5;
double v = 0.3 * speed_of_light;
double sigma = B * B / (4 * pi * massProton * 
electronConcentration * speed_of_light2);
\end{lstlisting}

Укажем для оптимизаторов количество параметров, ихи минимальные и максимальные значения и соответствие вектора параметров и физических величин. Оптимизируемыми параметрами являются - размер источника, магнетизация, доля заполнения и скорость расширения в первый момент времени, а так же показатели степени расширения со временем и изменения магнитного поля и концентрации с радиусом, то есть $\alpha, \beta, \gamma$ где эти величины определены через уравнения $R(t) = R_0 + \frac{1}{\alpha-1}\cdot V(0) \cdot t_0 \cdot ({t/t_0}^{\alpha-1}-1 )$, $B(R) = B(R_0)\cdot{R_0/R}^{\beta-1}$, $n(R) = n(R_0)\cdot{R_0/R}^{\gamma - 1}$. Единица добавлена к показателям степени для удобства численных расчетов при близости величин к нулю.
\begin{lstlisting}[language=c++]
const int Nparams = 8;
double minParameters[Nparams] = { 1E16, 0.0001, 0.01, 0.1, 
0.01 * speed_of_light, 1.1, 1.0, 1.0 };
double maxParameters[Nparams] = { 2E17, 1, 1000, 1.0, 0.6 * 
speed_of_light, 2.0, 3.5, 3.5 };
double vector[Nparams] = { rmax, sigma, electronConcentration, 
widthFraction, v, 2.0, 2.0, 3.0 };
for (int i = 0; i < Nparams; ++i) {
    vector[i] = vector[i] / maxParameters[i];
}
bool optPar[Nparams] = { true, true, true, true, true, true, true, true };
\end{lstlisting}

Далее создадим источник излучения. Воспользуемся моделью расширяющейся однородной сферической оболочки, с однородным магнитным полем, перпендикулярным лучу зрения и функцией распределения электронов, зависящей от угла между направлением магнитного поля и направлением расширения оболочки. Функции распределения получены с использованием Particle-in-Cell кода Smilei \cite{Derouillat} и содержатся в директории examplesData. Методика расчетов описана в статье \cite{BykovUnirse}. Количетсво распределений, посчитанных для углов от 0 до 90 градусов равно десяти. Их можно считать из соответствующих файлов, используя метод класса MassiveParticleDistributionFactory. Так же будет добавлено продолжение мтепенного хвоста, так как PIC расчеты не пользволяют получать длинные спектры из-за большой вычислительной сложности. Так же необходимо провести масштабирование распределения, так как в PIC расчетах испольовалось уменьшенное отношение масс протонов и электронов $m_p/m_e = 100$. Имея массив распределений создадим источник, учитывающий угловую зависимость, и передим его далее источнику. учитывающему зависимость от времени.

\begin{lstlisting}[language=c++]
const int Ndistributions = 10;

MassiveParticleIsotropicDistribution** angleDependentDistributions = 
MassiveParticleDistributionFactory::readTabulatedIsotropicDistributionsAddPowerLawTail
(massElectron, "./input/Ee", "./input/Fs", ".dat", 10, 
DistributionInputType::GAMMA_KIN_FGAMMA, electronConcentration, 200, 20 * me_c2, 3.5);
for (int i = 0; i < Ndistributions; ++i) {
(dynamic_cast<MassiveParticleTabulatedIsotropicDistribution*>
(angleDependentDistributions[i]))->rescaleDistribution(sqrt(18));
}

AngleDependentElectronsSphericalSource* angleDependentSource = new 
AngleDependentElectronsSphericalSource(20, 20, 4, Ndistributions, 
angleDependentDistributions,B,1.0,0,electronConcentration,rmax,0.5*rmax,distance);

RadiationTimeDependentSource* source = new 
ExpandingRemnantSource(rmax, B, electronConcentration, 0.3 * speed_of_light,
0.5, angleDependentSource, times[0]);

\end{lstlisting}

Теперь создадим вычислитель синхротронного излучения и два оптимизатора параметров - первый будет работать перебором параметров по сетке, а второй - градиентным спуском. Укажем количество точек по осям для перебора, количество итераций для градиентного спуска и диапазон энергий электронов, который будет рассматривать вычислитель синхротронного излучения.

\begin{lstlisting}[language=c++]
int Npoints[Nparams] = { 3, 3, 3, 3, 3, 3, 3, 3 };
int Niterations = 5;

double Emin = me_c2;
double Emax = 10000 * me_c2;

SynchrotronEvaluator* synchrotronEvaluator=new SynchrotronEvaluator(200, Emin, Emax);

RadiationTimeOptimizer* gridEnumOptimizer = 
new GridEnumRadiationTimeOptimizer(synchrotronEvaluator, minParameters, 
maxParameters, Nparams, Npoints);
RadiationTimeOptimizer* gradientOptimizer = 
new GradientDescentRadiationTimeOptimizer(synchrotronEvaluator,minParameters, 
maxParameters, Nparams, Niterations);
\end{lstlisting}

Применим созданые оптимизаторы и изменим параметры источника на найденные, соответствующие минимуму.

\begin{lstlisting}[language=c++]

gridEnumOptimizer->optimize(vector, optPar, energy, F, Error, Nenergy, Ntimes, times, source);

gradientOptimizer->optimize(vector, optPar, energy, F, Error, Nenergy, Ntimes, times, source);

source->resetParameters(vector, maxParameters);
\end{lstlisting}

Полученные в результате оптимизации парметры источника равны: радиус диска в начальный момент времени $R = 1.8\times10^17 \text{ см}$, магнитное поле $B = 1.6 \text{ Гс}$, концентрация электронов $n = 2.3 \text{ см}^{-3}$, доля толщины $fraction = 0.54 $, степени зависимости . Значение целевой функции $f \approx 50$. Модельный спектр излучения  с данными параметрами и наблюдательные данные изображены на рисунке \ref{synchrotronSeries}.

\begin{figure}
	\centering
	\includegraphics[width=12.5 cm]{./fig/synchrotronSeries.png} 
	\caption{Наблюдаемый и расчетный спектр радиоизлучения объекта CSS161010 на 99, 162 и 357 дни после вспышки}
	\label{synchrotronSeries}
\end{figure}
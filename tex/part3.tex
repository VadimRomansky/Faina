\chapter{Формулы расчета излучения}\label{Formulae}

\section{Преобразование функции распределения фотонов}
Функция распределения фотонов задана в сферических координатах $n_{ph}(\epsilon,\mu,\phi)$. Рассмотрим переход в систему отсчета, движущуюся в направлении оси $z$ с лоренц-фактором $\gamma = 1/\sqrt{1-\beta^2}$. Количество частиц в элементе фазового пространства $N$ - инвариант.
\begin{equation}
	N = n_{ph}(\epsilon,\mu,\phi)d\epsilon d\mu d\phi dV = n'_{ph}(\epsilon',\mu',\phi')d\epsilon' d\mu' d\phi' dV'
\end{equation}
Рассмотрим преобразование вектора четырех-импульса. Поперечные компоненты не изменяются, а временная и продольная меняются следющим образом, учитывая что $p_z = \mu \epsilon$:
\begin{equation}\label{lorentz_ph}
	\left(\begin{array}{c}
		\epsilon'\\
		\mu'\epsilon'
	\end{array}
	\right)
	= \left(
	\begin{array}{cc}
		\gamma & -\beta\gamma\\
		-\beta\gamma & \gamma
	\end{array}
	\right)
	\times
	\left(\begin{array}{c}
		\epsilon\\
		\mu\epsilon
	\end{array}
	\right)
\end{equation}
Из первой строчки матрицы получаем уравнение для допплеровского сдвига энергии
\begin{equation}\label{doppler_ph}
	\epsilon'=\gamma(1-\mu\beta)\epsilon
\end{equation}
Вычислим производные новой энергии по старым координатам
\begin{equation}
	\frac{d\epsilon'}{d\epsilon}=\gamma(1-\mu\beta)
\end{equation}
\begin{equation}
	\frac{d\epsilon'}{d\mu}=-\gamma\beta\epsilon
\end{equation}
Из второй строчки матрицы получаем $\mu'\epsilon'=-\beta\gamma\epsilon+\gamma\mu\epsilon$. Подставив значение $\epsilon'$ из \ref{doppler_ph} и сократив $\epsilon$ получим уравнение аберрации света
\begin{equation}\label{aberration_ph}
	\mu'=\frac{\mu-\beta}{1-\mu\beta}
\end{equation}
Заметим, что угол наклона луча в новой системе не зависит от энергии в старой системе. Вычислим частноую производную $\frac{d\mu'}{d\mu}$
\begin{equation}
	\frac{d\mu'}{d\mu}=\frac{d}{d\mu}\frac{1}{\beta}\frac{\beta\mu-1+1-\beta^2}{1-\mu\beta}=\frac{d}{d\mu}\frac{1}{\beta}\frac{1-\beta^2}{1-\mu\beta}=\frac{1-\beta^2}{(1-\mu\beta)^2}=\frac{1}{\gamma^2(1-\mu\beta)^2}
\end{equation}
Азиметальный угол не зависит от системы отсчета $\phi' = \phi$. Преобразование элемента объема описывается выражением $\frac{dV'}{dV} = \frac{\epsilon}{\epsilon'}$ см. ЛЛ Т2 параграф 10, вот только там используется переход в собственную систему. То есть
\begin{equation}
	\frac{dV'}{dV} = \frac{1}{\gamma(1-\mu\beta)}
\end{equation}
Матрица якоби преобразования координат выглядит следующим образом
\begin{equation}
	J=\left(
	\begin{array}{cccc}
		\frac{d\epsilon'}{d\epsilon} & \frac{d\epsilon'}{d\mu}& 0 & 0\\
		0 & \frac{d\mu'}{d\mu} & 0 & 0\\
		0 & 0 & 1 & 0\\
		0 & \frac{dV'}{d\mu} & 0 & \frac{dV'}{dV}
	\end{array}
	\right)
\end{equation}
При такой матрице якобиан, к счастью, равен произведению диагональных членов
\begin{equation}\label{jacobian_ph}
	\frac{D(\epsilon',\mu',\phi',V')}{D(\epsilon,\mu,\phi,V)}=\frac{d\epsilon'}{d\epsilon}\frac{d\mu'}{d\mu}\frac{dV'}{dV}=\gamma(1-\mu\beta)\frac{1}{\gamma^2(1-\mu\beta)^2}\frac{1}{\gamma(1-\mu\beta)}=\frac{1}{\gamma^2(1-\mu\beta)^2}
\end{equation}
И в итоге функция распределения фотонов преобразуется с помощью деления на вычисленный якобиан
\begin{equation}\label{distribution_ph}
	n'_{ph}(\epsilon',\mu',\phi') = \frac{n_{ph}(\epsilon,\mu,\phi)}{\frac{D(\epsilon',\mu',\phi',V')}{D(\epsilon,\mu,\phi,V)}}=\gamma^2(1-\mu\beta)^2 n_{ph}(\epsilon,\mu,\phi)
\end{equation}
\section{Комптоновское рассеяние}
Рассмотрим рассеяние фотонов на одном электроне. Сечение Клейна-Нишины в системе покоя электрона равно
\begin{equation}
	\frac{d\sigma}{d\epsilon_1'd\Omega_1'}=\frac{{r_e}^2}{2}\left(\frac{\epsilon_1'}{\epsilon_0'}\right)^2\left(\frac{\epsilon_1'}{\epsilon_0'}+\frac{\epsilon_0'}{\epsilon_1'}-sin^2\Theta'\right)
\end{equation}
Где $r_e$ - классический радиус электрона, $\epsilon_0'$ и $\epsilon_1'$ - энергии начального и конечного фотона, соответственно, $\Theta'$ - угол между начальным и конечным фотоном. Штрихованные индексы относятся к системе отсчета электрона.
Число фотонов, 

\section{Синхротронное излучение}
Процесс синхротронного излучения хороши известен и описан в классических работах. Но с точки зрения квантовой электродинамки, любому процессу излучения можно так же сопоставить процесс поглощения. Сечение процесса синхротронного самопоглощения описано в работе Гизеллини и Свенсона \cite{Ghisellini1991}. Спектральная плотность мощности излучения единицы объема вещества определеяется формулой
\begin{equation} \label{emission}
	I(\nu)=\int_{E_{min}}^{E_{max}} dE \frac {\sqrt {3}{e}^{3}n F(E) B \sin ( \phi)}{{m_e}{c}^{2}}
	\frac{\nu}{\nu_c}\int_{\frac {\nu}{\nu_c}}^{\infty }\it K_{5/3}(x)dx,
\end{equation}
где $\phi$ это угол межде вектором магнитного поля и лучом зрения, $\displaystyle\nu_{c}$ критическая частота, определяемая выражением $\displaystyle\nu_{c} = 3 e^{2} B \sin(\phi) E^{2}/4\pi {m_{e}}^{3} c^{5}$, и~$K_{5/3}$ - функция МакДональда.
Коэффициент поглощения для фотонов, распростроняющихся вдоль луча зрения равен
\begin{equation}\label{absorption}
	k(\nu)=\int_{E_{min}}^{E_{max}}dE\frac {\sqrt {3}{e}^{3}}{8\pi m_e \nu^2}\frac{n B\sin(\phi)}{E^2}
	\frac{d}{dE} E^2 F(E)\frac {\nu}{ \nu_c}\int_{\frac {\nu}{ \nu_c}}^{\infty }K_{5/3}(x) dx.
\end{equation}
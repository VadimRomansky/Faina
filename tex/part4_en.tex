\chapter{Формулы расчета излучения}\label{Formulae}

\section{Преобразование функции распределения фотонов}
Функция распределения фотонов задана в сферических координатах $n_{ph}(\epsilon,\mu,\phi)$. Рассмотрим переход в систему отсчета, движущуюся в направлении оси $z$ с лоренц-фактором $\gamma = 1/\sqrt{1-\beta^2}$. Количество частиц в элементе фазового пространства $N$ - инвариант.
\begin{equation}
	N = n_{ph}(\epsilon,\mu,\phi)d\epsilon d\mu d\phi dV = n'_{ph}(\epsilon',\mu',\phi')d\epsilon' d\mu' d\phi' dV'
\end{equation}
Рассмотрим преобразование вектора четырех-импульса. Поперечные компоненты не изменяются, а временная и продольная меняются следющим образом, учитывая что $p_z = \mu \epsilon$:
\begin{equation}\label{lorentz_ph}
	\left(\begin{array}{c}
		\epsilon'\\
		\mu'\epsilon'
	\end{array}
	\right)
	= \left(
	\begin{array}{cc}
		\gamma & -\beta\gamma\\
		-\beta\gamma & \gamma
	\end{array}
	\right)
	\times
	\left(\begin{array}{c}
		\epsilon\\
		\mu\epsilon
	\end{array}
	\right)
\end{equation}
Из первой строчки матрицы получаем уравнение для допплеровского сдвига энергии
\begin{equation}\label{doppler_ph}
	\epsilon'=\gamma(1-\mu\beta)\epsilon
\end{equation}
Вычислим производные новой энергии по старым координатам
\begin{equation}
	\frac{d\epsilon'}{d\epsilon}=\gamma(1-\mu\beta)
\end{equation}
\begin{equation}
	\frac{d\epsilon'}{d\mu}=-\gamma\beta\epsilon
\end{equation}
Из второй строчки матрицы получаем $\mu'\epsilon'=-\beta\gamma\epsilon+\gamma\mu\epsilon$. Подставив значение $\epsilon'$ из \ref{doppler_ph} и сократив $\epsilon$ получим уравнение аберрации света
\begin{equation}\label{aberration_ph}
	\mu'=\frac{\mu-\beta}{1-\mu\beta}
\end{equation}
Заметим, что угол наклона луча в новой системе не зависит от энергии в старой системе. Вычислим частноую производную $\frac{d\mu'}{d\mu}$
\begin{equation}
	\frac{d\mu'}{d\mu}=\frac{d}{d\mu}\frac{1}{\beta}\frac{\beta\mu-1+1-\beta^2}{1-\mu\beta}=\frac{d}{d\mu}\frac{1}{\beta}\frac{1-\beta^2}{1-\mu\beta}=\frac{1-\beta^2}{(1-\mu\beta)^2}=\frac{1}{\gamma^2(1-\mu\beta)^2}
\end{equation}
Азимутальный угол не зависит от системы отсчета $\phi' = \phi$. Преобразование элемента объема описывается выражением $\frac{dV'}{dV} = \frac{\epsilon}{\epsilon'}$ см. ЛЛ Т2 параграф 10, вот только там используется переход в собственную систему. То есть
\begin{equation}
	\frac{dV'}{dV} = \frac{1}{\gamma(1-\mu\beta)}
\end{equation}
Матрица якоби преобразования координат выглядит следующим образом
\begin{equation}
	J=\left(
	\begin{array}{cccc}
		\frac{d\epsilon'}{d\epsilon} & \frac{d\epsilon'}{d\mu}& 0 & 0\\
		0 & \frac{d\mu'}{d\mu} & 0 & 0\\
		0 & 0 & 1 & 0\\
		0 & \frac{dV'}{d\mu} & 0 & \frac{dV'}{dV}
	\end{array}
	\right)
\end{equation}
При такой матрице якобиан, к счастью, равен произведению диагональных членов
\begin{equation}\label{jacobian_ph}
	\frac{D(\epsilon',\mu',\phi',V')}{D(\epsilon,\mu,\phi,V)}=\frac{d\epsilon'}{d\epsilon}\frac{d\mu'}{d\mu}\frac{dV'}{dV}=\gamma(1-\mu\beta)\frac{1}{\gamma^2(1-\mu\beta)^2}\frac{1}{\gamma(1-\mu\beta)}=\frac{1}{\gamma^2(1-\mu\beta)^2}
\end{equation}
И в итоге функция распределения фотонов преобразуется с помощью деления на вычисленный якобиан
\begin{equation}\label{distribution_ph}
	n'_{ph}(\epsilon',\mu',\phi') = \frac{n_{ph}(\epsilon,\mu,\phi)}{\frac{D(\epsilon',\mu',\phi',V')}{D(\epsilon,\mu,\phi,V)}}=\gamma^2(1-\mu\beta)^2 n_{ph}(\epsilon,\mu,\phi)
\end{equation}
\section{Комптоновское рассеяние}\label{comptonFormulaSection}
Рассмотрим рассеяние фотонов на одном электроне, движущемся вдоль ось z, см \cite{Dubus}. Сечение Клейна-Нишины в системе покоя электрона равно
\begin{equation}
	\frac{d\sigma}{d\epsilon_1'd\Omega_1'}=\frac{{r_e}^2}{2}\left(\frac{\epsilon_1'}{\epsilon_0'}\right)^2(\frac{\epsilon_1'}{\epsilon_0'}+\frac{\epsilon_0'}{\epsilon_1'}-\sin^2\Theta') \delta(\epsilon_1' - \frac{\epsilon_0'}{1+\frac{\epsilon_0'}{m_e c^2}(1 - \cos \Theta')})
\end{equation}
Где $r_e$ - классический радиус электрона, $\epsilon_0'$ и $\epsilon_1'$ - энергии начального и конечного фотона, соответственно, $\Theta'$ - угол между начальным и конечным фотоном, определяемый выражением $\cos\Theta' =\cos \theta_0' \cos \theta_1' + \sin \theta_0' \sin \theta_1' \cos(\phi_1' - \phi_0')$. Штрихованные индексы относятся к системе отсчета электрона. При этом начальная и конечная энергии фотонов оказываются связаны соотношениями
\begin{equation}
	\epsilon_1'=\frac{\epsilon_0'}{1+\frac{\epsilon_0'}{m_e c^2}(1 - \cos \Theta')}	
\end{equation}
\begin{equation}
	\epsilon_0'=\frac{\epsilon_1'}{1-\frac{\epsilon_1'}{m_e c^2}(1 - \cos \Theta')}
\end{equation}
Число фотонов, рассеявшихся в заданный телесный угол в единицу времени в промежуток энергии в системе покоя электрона равно
\begin{equation}
\frac{dN'}{dt'd\epsilon_1'd\Omega_1'}=\int c \frac{d\sigma}{d\epsilon_1'd\Omega_1'} \frac{dn'}{d\epsilon_0'd\Omega_0'}d\Omega_0'd\epsilon_0'
\end{equation}

Перепишем дельта-функцию через энергию начального фотона с помощью соотношения 
\begin{equation}
	\delta(f(x)) = \sum \frac{\delta(x-x_k)}{|f'(x_k)|}
\end{equation}
где $x_k$ - корни функции $f(x)$. Производная выражения внутри дельта-функции равна
\begin{equation}
	\frac{d\epsilon_1'}{d\epsilon_0'}=\frac{1}{(1+\frac{\epsilon_0'}{m_e c^2}(1 - \cos \Theta'))^2}
\end{equation}
и она сократится с квадратом отношения энергий в формуле для сечения. Функцию распределения начальных фотонов выразим в лабораторной системе с помощью выражения \ref{distribution_ph}.
\begin{equation}
	\frac{dN'}{dt'd\epsilon_1'd\Omega_1'}=\int \frac{r_e^2 c}{2} \gamma_e^2 (1 - \mu_0 \beta_e)^2 (\frac{\epsilon_1'}{\epsilon_0'}+\frac{\epsilon_0'}{\epsilon_1'}-\sin^2\Theta')\frac{dn}{d\epsilon_0 d\Omega_0} \delta(\epsilon_0' - \frac{\epsilon_1'}{1-\frac{\epsilon_1'}{m_e c^2}(1 - \cos \Theta')}) d\epsilon_0'd\mu_0' d\phi_0'
\end{equation}
Теперь избавимся от дельта-функции, проинтегрировав по $\epsilon_0'$.
\begin{equation}
	\frac{dN'}{dt'd\epsilon_1'd\Omega_1'}=\int \frac{r_e^2 c}{2} \gamma_e^2 (1 - \mu_0 \beta_e)^2 (1 + \cos^2\Theta'+(\frac{\epsilon_1'}{m_e c^2})^2\frac{(1-\cos\Theta')^2}{1-\frac{\epsilon_1'}{m_e c^2}(1 - \cos \Theta')})\frac{dn}{d\epsilon_0 d\Omega_0}d\mu_0' d\phi_0'
\end{equation}
Осталось перевести поток рассеяных фотонов в лабораторную систему отсчета $\frac{dN}{dt d\epsilon_1 d\Omega_1} = \frac{dN'}{dt' d\epsilon_1' d\Omega_1'}\frac{dt'}{dt}\frac{d\epsilon_1'}{d\epsilon_1}\frac{d\Omega_1'}{d\Omega_1}$. Используя то, что $dt = \gamma_e dt'$, $\epsilon = \frac{1}{\gamma_e(1 -\mu_1\beta_e)}\epsilon'$ и $\mu_1' = \frac{\mu_1-\beta_e}{1-\mu_1 \beta_e}$ получим
\begin{equation} \label{compton_elframe}
	\frac{dN}{dt d\epsilon_1 d\Omega_1}=\int \frac{r_e^2 c}{2} \frac{(1 - \mu_0 \beta_e)^2}{1-\mu_1\beta_e} (1 + \cos^2\Theta'+(\frac{\epsilon_1'}{m_e c^2})^2\frac{(1-\cos\Theta')^2}{1-\frac{\epsilon_1'}{m_e c^2}(1 - \cos \Theta')})\frac{dn}{d\epsilon_0 d\Omega_0}d\mu_0' d\phi_0'	
\end{equation}
При интегрировании нужно выразить углы в лабораторной системе отсчета $\mu_0, \phi_0$ через переменные интегрирования $\mu_0', \phi_0'$. Для расчета рассеяния на распределении электронов нужно проинтегрировать формулу \ref{compton_elframe} с функцией распределения электронов, нормированной на количество частиц. При этом надо учесть разные направления движения электронов и произвести повороты углов.

Так же может быть удобно интегрировать в переменных лабораторной системы расчета, тогда выражение для потока фотонов будет следующим
\begin{equation}\label{compton_labframe}
	\frac{dN}{dt d\epsilon_1 d\Omega_1}=\int \frac{r_e^2 c}{2} \frac{1}{\gamma_e^2(1-\mu_1\beta_e)} (1 + \cos^2\Theta'+(\frac{\epsilon_1'}{m_e c^2})^2\frac{(1-\cos\Theta')^2}{1-\frac{\epsilon_1'}{m_e c^2}(1 - \cos \Theta')})\frac{dn}{d\epsilon_0 d\Omega_0}d\mu_0 d\phi_0
\end{equation}
При рассмотрении процессов, связанных с электронами высоких энергий $\gamma_e \approx 10^8$ относительные численные погрешности вычислений могут быть очень велики, так как $\beta_e$ и $\mu_0, \mu_1, \cos \Theta'$ оказываются слишком близки к единице и стандартный тип double может не разрешать это отличие. Поэтому для численных вычислений оказывается полезным ввести следующие вспомогательные величины:
\begin{equation}
	\delta_e = 1 - \beta_e
\end{equation}
\begin{equation}
	\text{versin}~\theta = 1 - \cos \theta
\end{equation}
Тогда выражения вида $1 - \mu \beta_e$ в этих величинах перепишется как
\begin{equation}
	1 - \mu \beta_e =\text{versin}~\theta + \delta_e - \text{versin}~\theta~\delta_e
\end{equation}
а выражение для угла между конечным и начальным фотоном как
\begin{equation}
	1 - \cos \Theta' = \text{versin}~\theta_0' + \text{versin}~\theta_1' - \text{versin}~ \theta_0' \text{versin}~\theta_1' - \sin \theta_0'\sin \theta_1' \cos(\phi_1'-\phi_0')
\end{equation}
С использованием данных выражений значительно повышается точность и максимальные доступные к рассмотрению энергии фотонов и электронов.

В случае изотропных функций распределения фотонов и релятивистских электронов можно произвести аналитическое интегрирование по угловым переменным \cite{JonesCompton, BykovUvarov2000}, и тогда для вычисления излучения достаточно лишь провести интегрирования по энергиям по формуле
\begin{equation}
	\frac{dN}{dt d\epsilon_1 d\Omega_1}=\int \frac{2 \pi r_e^2 m_e c^3 }{\epsilon_0 \gamma_e^2} \frac{dn_{ph}}{d\epsilon_0}\frac{dn_e}{d\epsilon_e}(2 q~ \ln(q)+1+q-2q^2+\frac{q^2(1-q)\Gamma^2}{2(1+q\Gamma)})d\epsilon_0d\epsilon_e
\end{equation}
где $\Gamma=4\epsilon_0\gamma_e/m_e c^2$, $q=\epsilon_1/((\gamma_e m_e c^2-\epsilon_1)\Gamma)$.
\section{Синхротронное излучение}\label{synchrotronFormulaSection}
Процесс синхротронного излучения хороши известен и описан в классических работах. Но с точки зрения квантовой электродинамки, любому процессу излучения можно так же сопоставить процесс поглощения. Сечение процесса синхротронного самопоглощения описано в работе Гизеллини и Свенсона \cite{Ghisellini1991}. Спектральная плотность мощности излучения единицы объема вещества определеяется формулой
\begin{equation} \label{emission}
	I(\nu)=\int_{E_{min}}^{E_{max}} dE \frac {\sqrt {3}{e}^{3}n F(E) B \sin ( \phi)}{{m_e}{c}^{2}}
	\frac{\nu}{\nu_c}\int_{\frac {\nu}{\nu_c}}^{\infty }\it K_{5/3}(x)dx,
\end{equation}
где $\phi$ это угол межде вектором магнитного поля и лучом зрения, $\displaystyle\nu_{c}$ критическая частота, определяемая выражением $\displaystyle\nu_{c} = 3 e^{2} B \sin(\phi) E^{2}/4\pi {m_{e}}^{3} c^{5}$, и~$K_{5/3}$ - функция МакДональда.
Коэффициент поглощения для фотонов, распростроняющихся вдоль луча зрения равен
\begin{equation}\label{absorption}
	k(\nu)=\int_{E_{min}}^{E_{max}}dE\frac {\sqrt {3}{e}^{3}}{8\pi m_e \nu^2}\frac{n B\sin(\phi)}{E^2}
	\frac{d}{dE} E^2 F(E)\frac {\nu}{ \nu_c}\int_{\frac {\nu}{ \nu_c}}^{\infty }K_{5/3}(x) dx.
\end{equation}
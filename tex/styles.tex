%%% Макет страницы %%%
\geometry{a4paper,top=2cm,bottom=2cm,left=3cm,right=1cm}

\titleformat{\chapter}[display]
  {\normalfont\Large\filcenter\bfseries}{\chaptertitlename\ \thechapter}{20pt}{\LARGE}
\titlespacing*{\chapter}
  {0pt}{30pt}{20pt}

% см. http://tex.stackexchange.com/questions/134031/how-to-adjust-the-size-and-placement-of-chapter-heading-in-report-class
% и http://tex.stackexchange.com/questions/59726/change-size-of-section-subsection-subsubsection-paragraph-and-subparagraph-ti

%%% Кодировки и шрифты %%%
%\renewcommand{\rmdefault}{ftm} % Включаем Times New Roman

%%% Выравнивание и переносы %%%
\sloppy					% Избавляемся от переполнений
\clubpenalty=10000		% Запрещаем разрыв страницы после первой строки абзаца
\widowpenalty=10000		% Запрещаем разрыв страницы после последней строки абзаца
\linespread{1.3}

%%% Библиография %%%
\makeatletter
\bibliographystyle{ugost2008ns_v1.bst}	% Оформляем библиографию в соответствии с ГОСТ 7.0.5
%\bibliographystyle{utf8gost705u}
%\bibliographystyle{ugost2008}
\renewcommand{\@biblabel}[1]{#1.}	% Заменяем библиографию с квадратных скобок на точку:
\makeatother

%%% Изображения %%%
\graphicspath{{img_part1/}{img_part2/}{img_part3/}{img_part4/}{img_part5/}} % Пути к изображениям

%%% Цвета гиперссылок %%%
\definecolor{linkcolor}{rgb}{0.9,0,0}
\definecolor{citecolor}{rgb}{0,0,1}
\definecolor{urlcolor}{rgb}{0,0,1}
\hypersetup{
    colorlinks, linkcolor={linkcolor},
    citecolor={citecolor}, urlcolor={urlcolor}
}

%%% Оглавление %%%
\renewcommand{\cftchapdotsep}{\cftdotsep}


\newcommand{\kws}{\textit{KW}\ }
\newcommand{\rmn}[1]{{\mathrm{#1}}}
\newcommand\fdg{\mbox{$.\!\!^\circ$}}%
\newcommand\farcm{\mbox{$.\mkern-4mu^\prime$}}%
\newcommand\farcs{\mbox{$.\!\!^{\prime\prime}$}}%
\newcommand\nodata{ ~$\cdots$~ }%

